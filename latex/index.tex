\href{https://travis-ci.org/Erriez/ErriezBH1750}{\tt }

This is a 16-\/bit \hyperlink{class_b_h1750}{B\+H1750} digital ambient light sensor on a G\+Y-\/302 breakout P\+CB\+:



\subsection*{Arduino library features}


\begin{DoxyItemize}
\item Measurement in L\+UX
\item Three operation modes\+:
\begin{DoxyItemize}
\item Continues conversion
\item One-\/time conversion
\end{DoxyItemize}
\item Three selectable resolutions\+:
\begin{DoxyItemize}
\item Low 4 L\+UX resolution (low power)
\item High 1 L\+UX resolution
\item High 0.\+5 L\+UX resolution
\end{DoxyItemize}
\item Asynchronous and synchronous conversion
\end{DoxyItemize}

\subsection*{\hyperlink{class_b_h1750}{B\+H1750} sensor specifications}


\begin{DoxyItemize}
\item Operating voltage\+: 3.\+3V .. 4.\+5V max
\item Low current by power down\+: max 1uA
\item I2C bus interface\+: max 400k\+Hz
\item Ambience light\+:
\begin{DoxyItemize}
\item Range\+: 1 -\/ 65535 lx
\item Deviation\+: +/-\/ 20\%
\item Selectable resolutions\+:
\begin{DoxyItemize}
\item 4 lx (low resolution, max 24 ms measurement time)
\item 1 lx (mid resolution max 180 ms measurement time)
\item 0.\+5 lx (high resolution 180 ms measurement time)
\end{DoxyItemize}
\end{DoxyItemize}
\item No additional electronic components needed
\end{DoxyItemize}

\subsection*{G\+Y-\/302 breakout specifications}


\begin{DoxyItemize}
\item Supply voltage\+: 3.\+3 .. 5V
\item 5V tolerant I2C S\+CL and S\+DA pins
\item 2 selectable I2C addresses with A\+D\+DR pin high or low/floating
\end{DoxyItemize}

\subsection*{Hardware}



{\bfseries Connection Arduino U\+NO board -\/ \hyperlink{class_b_h1750}{B\+H1750}}

\tabulinesep=1mm
\begin{longtabu} spread 0pt [c]{*5{|X[-1]}|}
\hline
\rowcolor{\tableheadbgcolor}{\bf Pins board -\/ \hyperlink{class_b_h1750}{B\+H1750} }&\PBS\centering {\bf V\+CC }&\PBS\centering {\bf G\+ND }&\PBS\centering {\bf S\+DA }&\PBS\centering {\bf S\+CL  }\\\cline{1-5}
\endfirsthead
\hline
\endfoot
\hline
\rowcolor{\tableheadbgcolor}{\bf Pins board -\/ \hyperlink{class_b_h1750}{B\+H1750} }&\PBS\centering {\bf V\+CC }&\PBS\centering {\bf G\+ND }&\PBS\centering {\bf S\+DA }&\PBS\centering {\bf S\+CL  }\\\cline{1-5}
\endhead
Arduino U\+NO (A\+T\+Mega328 boards) &\PBS\centering 5V &\PBS\centering G\+ND &\PBS\centering A4 &\PBS\centering A5 \\\cline{1-5}
Arduino Mega2560 &\PBS\centering 5V &\PBS\centering G\+ND &\PBS\centering D20 &\PBS\centering D21 \\\cline{1-5}
Arduino Leonardo &\PBS\centering 5V &\PBS\centering G\+ND &\PBS\centering D2 &\PBS\centering D3 \\\cline{1-5}
Arduino D\+UE (A\+T\+S\+A\+M3\+X8E) &\PBS\centering 3\+V3 &\PBS\centering G\+ND &\PBS\centering 20 &\PBS\centering 21 \\\cline{1-5}
E\+S\+P8266 &\PBS\centering 3\+V3 &\PBS\centering G\+ND &\PBS\centering G\+P\+I\+O4 (D2) &\PBS\centering G\+P\+I\+O5 (D1) \\\cline{1-5}
E\+S\+P32 &\PBS\centering 3\+V3 &\PBS\centering G\+ND &\PBS\centering G\+P\+I\+O21 &\PBS\centering G\+P\+I\+O22 \\\cline{1-5}
\end{longtabu}
Note\+: Tested E\+S\+P8266 / E\+S\+P32 boards\+:


\begin{DoxyItemize}
\item {\bfseries E\+S\+P8266 boards}\+: E\+S\+P12E / We\+Mos D1 \& R2 / Node M\+CU v2 / v3
\item {\bfseries E\+S\+P32 boards\+:} We\+Mos L\+O\+L\+I\+N32 / L\+O\+L\+IN D32
\end{DoxyItemize}

Other unlisted M\+CU\textquotesingle{}s may work, but are not tested.

\paragraph*{We\+Mos L\+O\+L\+I\+N32 with O\+L\+ED display}

Change the following Wire initialization to\+:


\begin{DoxyCode}
1 \{c++\}
2 // WeMos LOLIN32 with OLED support
3 Wire.begin(5, 4);
\end{DoxyCode}


\subsubsection*{I2C address}


\begin{DoxyItemize}
\item {\ttfamily A\+D\+DR} pin {\ttfamily L\+OW} for I2C address 0x23 (0x46 including R/W bit)
\item {\ttfamily A\+D\+DR} pin {\ttfamily H\+I\+GH} for I2C address 0x5C (0x\+B8 including R/W bit)
\end{DoxyItemize}

{\bfseries Note\+:} {\ttfamily A\+D\+DR} pin may be floating (open) which is the same as L\+OW.

\subsection*{Examples}

Examples $\vert$ Erriez \hyperlink{class_b_h1750}{B\+H1750}\+:


\begin{DoxyItemize}
\item Continues\+Mode $\vert$ \href{https://github.com/Erriez/ErriezBH1750/blob/master/examples/ContinuesMode/BH1750ContinuesAsynchronous/BH1750ContinuesAsynchronous.ino}{\tt B\+H1750\+Continues\+Asynchronous}
\item Continues\+Mode $\vert$ \href{https://github.com/Erriez/ErriezBH1750/blob/master/examples/ContinuesMode/BH1750ContinuesBasic/BH1750ContinuesBasic.ino}{\tt B\+H1750\+Continues\+Basic}
\item Continues\+Mode $\vert$ \href{https://github.com/Erriez/ErriezBH1750/blob/master/examples/ContinuesMode/BH1750ContinuesHighResolution/BH1750ContinuesHighResolution.ino}{\tt B\+H1750\+Continues\+High\+Resolution}
\item Continues\+Mode $\vert$ \href{https://github.com/Erriez/ErriezBH1750/blob/master/examples/ContinuesMode/BH1750ContinuesLowResolution/BH1750ContinuesLowResolution.ino}{\tt B\+H1750\+Continues\+Low\+Resolution}
\item Continues\+Mode $\vert$ \href{https://github.com/Erriez/ErriezBH1750/blob/master/examples/ContinuesMode/BH1750ContinuesPowerMgt/BH1750ContinuesPowerMgt.ino}{\tt B\+H1750\+Continues\+Power\+Mgt}
\item One\+Time\+Mode $\vert$ \href{https://github.com/Erriez/ErriezBH1750/blob/master/examples/OneTimeMode/BH1750OneTimeBasic/BH1750OneTimeBasic.ino}{\tt B\+H1750\+One\+Time\+Basic}
\item One\+Time\+Mode$\vert$ \href{https://github.com/Erriez/ErriezBH1750/blob/master/examples/OneTimeMode/BH1750OneTimeHighResolution/BH1750OneTimeHighResolution.ino}{\tt B\+H1750\+One\+Time\+High\+Resolution}
\item One\+Time\+Mode$\vert$ \href{https://github.com/Erriez/ErriezBH1750/blob/master/examples/OneTimeMode/BH1750OneTimeLowResolution/BH1750OneTimeLowResolution.ino}{\tt B\+H1750\+One\+Time\+Low\+Resolution}
\item One\+Time\+Mode$\vert$ \href{https://github.com/Erriez/ErriezBH1750/blob/master/examples/OneTimeMode/BH1750OneTimePowerMgt/BH1750OneTimePowerMgt.ino}{\tt B\+H1750\+One\+Time\+Power\+Mgt}
\end{DoxyItemize}

\subsection*{Documentation}


\begin{DoxyItemize}
\item \href{https://erriez.github.io/ErriezBH1750}{\tt Doxygen online H\+T\+ML}
\item \href{https://github.com/Erriez/ErriezBH1750/raw/gh-pages/latex/ErriezBH1750.pdf}{\tt Doxygen P\+DF}
\item \href{https://github.com/Erriez/ErriezBH1750/raw/master/extras/BH1750_datasheet.pdf}{\tt B\+H1750 chip datasheet}
\end{DoxyItemize}

\subsection*{Example continues conversion high resolution}


\begin{DoxyCode}
1 \{c++\}
2 #include <Wire.h>
3 #include <ErriezBH1750.h>
4 
5 // ADDR line LOW/open:  I2C address 0x23 (0x46 including R/W bit) [default]
6 // ADDR line HIGH:      I2C address 0x5C (0xB8 including R/W bit)
7 BH1750 sensor(LOW);
8 
9 void setup()
10 \{
11   Serial.begin(115200);
12   Serial.println(F("BH1750 continues measurement high resolution example"));
13 
14   // Initialize I2C bus
15   Wire.begin();
16 
17   // Initialize sensor in continues mode, high 0.5 lx resolution
18   sensor.begin(ModeContinuous, ResolutionHigh);
19 
20   // Start conversion
21   sensor.startConversion();
22 \}
23 
24 void loop()
25 \{
26   uint16\_t lux;
27 
28   // Wait for completion (blocking busy-wait delay)
29   if (sensor.isConversionCompleted()) \{
30     // Read light
31     lux = sensor.read();
32 
33     // Print light
34     Serial.print(F("Light: "));
35     Serial.print(lux / 2);
36     Serial.print(F("."));
37     Serial.print(lux % 10);
38     Serial.println(F(" LUX"));
39   \}
40 \}
\end{DoxyCode}
 {\bfseries Output} 
\begin{DoxyCode}
1 \{c++\}
2 BH1750 continues measurement high resolution example
3 Light: 15.0 LUX
4 Light: 31.2 LUX
5 Light: 385.0 LUX
6 Light: 575.1 LUX
7 Light: 667.5 LUX
\end{DoxyCode}


\subsection*{Usage}

\subsubsection*{Initialization}


\begin{DoxyCode}
1 \{c++\}
2 #include <Wire.h>
3 #include <ErriezBH1750.h>
4 
5 // ADDR line LOW/open:  I2C address 0x23 (0x46 including R/W bit) [default]
6 // ADDR line HIGH:      I2C address 0x5C (0xB8 including R/W bit)
7 BH1750 sensor(LOW);
8 
9 void setup()
10 \{
11     // Initialize I2C bus
12     Wire.begin();
13 
14     // Initialize sensor with a mode and resolution:
15     //   Modes:
16     //     ModeContinuous
17     //     ModeOneTime
18     //   Resolutions:
19     //     ResolutionLow (4 lx resolution)
20     //     ResolutionMid (1 lx resolution)
21     //     ResolutionHigh (0.5 lx resolution)
22     sensor.begin(mode, resolution);
23 \}
\end{DoxyCode}


\subsubsection*{Start conversion}


\begin{DoxyCode}
1 \{Wire.begin();```\}
2 
3 ```c++
4 sensor.startConversion();
\end{DoxyCode}


\subsubsection*{Wait for completion asynchronous (non-\/blocking)}

The sensor conversion completion status can be checked asynchronously before reading the light value\+:


\begin{DoxyCode}
1 \{c++\}
2 bool completed = sensor.isConversionCompleted();
\end{DoxyCode}


\subsubsection*{Wait for completion synchronous (blocking)}

The sensor conversion completion status can be checked synchronously before reading the light value\+:


\begin{DoxyCode}
1 \{c++\}
2 // Wait for completion
3 // completed = false: Timeout or device in power-down
4 bool completed = sensor.waitForCompletion();
\end{DoxyCode}


\subsubsection*{Read light value in L\+UX}

{\bfseries One-\/time mode\+:} The application must wait or check for a completed conversion, otherwise the sensor may return an invalid value. {\bfseries Continues mode\+:} The application can call this function without checking completion, but is not recommended when accurate values are required.

Read sensor light value\+:


\begin{DoxyCode}
1 \{c++\}
2 // lux = 0: No light or not initialized
3 uint16\_t lux = sensor.read();
\end{DoxyCode}
 For 4 lx low and 1 lx high resolutions\+:


\begin{DoxyCode}
1 \{c++\}
2 // Print low and medium resolutions
3 Serial.print(F("Light: "));
4 Serial.print(lux);
5 Serial.println(F(" LUX"));
\end{DoxyCode}


For 0.\+5 lx high resolution\+:


\begin{DoxyCode}
1 \{c++\}
2 // Print high resolution
3 Serial.print(F("Light: "));
4 Serial.print(lux / 2);
5 Serial.print(F("."));
6 Serial.print(lux % 10);
7 Serial.println(F(" LUX"));
\end{DoxyCode}


\subsubsection*{Power down}

The device enters power down automatically after a one-\/time conversion.

A manual power-\/down in continues mode can be generated by calling\+:


\begin{DoxyCode}
1 \{c++\}
2 sensor.powerDown();
\end{DoxyCode}


\subsection*{Library dependencies}


\begin{DoxyItemize}
\item Built-\/in {\ttfamily Wire.\+h}
\end{DoxyItemize}

\subsection*{Library installation}

Please refer to the \href{https://github.com/Erriez/ErriezArduinoLibrariesAndSketches/wiki}{\tt Wiki} page.

\subsection*{Other Arduino Libraries and Sketches from Erriez}


\begin{DoxyItemize}
\item \href{https://github.com/Erriez/ErriezArduinoLibrariesAndSketches}{\tt Erriez Libraries and Sketches} 
\end{DoxyItemize}